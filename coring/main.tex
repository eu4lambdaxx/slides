\documentclass{beamer}

\usepackage{hyperref}
\hypersetup{
  colorlinks=true,
  linkcolor=blue,
}
\usepackage{amsmath}


% formatting (normal mode)
\newcommand{\idea}[1]{\textbf{\textcolor{blue}{#1}}}

\title{Coring Rules}
\subtitle{v1.30.3}
\author{$\lambda x. x$}
\date{\today}

\begin{document}

\begin{frame}
  \titlepage
\end{frame}

\begin{frame}
  \frametitle{Disclaimer}
  This document is meant to still be useful standalone, so I am not following good slide guidelines and instead being overly verbose. That being said, a corresponding video with live examples is available on my \href{https://www.youtube.com/watch?v=dQw4w9WgXcQ}{youtube channel}.
\end{frame}

\begin{frame}
  \frametitle{The tl;dr}
  \begin{itemize}
    \item Land connection \textbf{control} from \textbf{controlled} capital
    \item Adjacent to (owned/unowned) core
    \item Same continent next to vassal owned core
    \item Coastal from \textbf{owned} and \textbf{controlled} coastal core (based on colonial range)
    \item Land connection \textbf{ownership} from \textbf{owned} and \textbf{controlled} coastal core (based on colonial range)
    \item Colonial nation (CN) spaghetti
  \end{itemize}
\end{frame}

\begin{frame}
  \frametitle{Owner vs Control}
  \begin{itemize}
    \item Owner merely requires owning the province.
    \item Control requires either ownership without occupation (so for example, during peace), or occupation of non-owned (enemy) territory.
    \item The two are not mutually exclusive.
  \end{itemize}
\end{frame}

\begin{frame}
  \frametitle{The Capital Rule}
  \begin{itemize}
    \item ``You can core any province that is connected to your capital through \textbf{control}''
    \item Your capital cannot be occupied.
    \item Occupations do count since it's control!
  \end{itemize}
\end{frame}

\begin{frame}
  \frametitle{Generic Core Rule}
  \begin{itemize}
    \item ``You can core any province that is next to your core''
    \item Whether the core is controlled/owned is irrelevant.
  \end{itemize}
\end{frame}

\begin{frame}
  \frametitle{Vassal Rule}
  \begin{itemize}
    \item ``You can core any same-continent province that is adjacent to an \textbf{owned} subject core''
    \item All subjects count\footnote{I have not verified if CNs count. They have special rules.}.
    \item The continent of your vassal is irrelevant.
    \item The continent of the \textbf{owned} subject core is also irrelevant.
  \end{itemize}
\end{frame}

\begin{frame}
  \frametitle{\textbf{Owned} and \textbf{Controlled} Coastal Cores} 
  \begin{itemize}
    \item ``All coastal cores exert a distance based coring range both via sea and by land.``
    \item Note that here the coastal core has to be \textbf{owned} and \textbf{controlled}.
    \item The easiest way to see the land based coring range is to integrate Dai Viet and annex Ming as a far-away nation, say Ottomans.
    \item Land based coring range requires land connection via \textbf{ownership}, so occupations neither help nor hurt you.
    \item Yes, colonial range helps the land coring range as well. In particular, raising dip tech does allow you to core all of Ming with Dai Viet cores.
    \item Since incomplete colonies count as uncored, they do not exert coastal colonial range.
  \end{itemize}
\end{frame}

\begin{frame}
  \frametitle{Colonial Spaghetti}
  \begin{itemize}
    \item Having a CN unlocks more coring options. This is not fully investigated/known, so I'm open to any corrections!
    \item ``A \textbf{controlled} coastal province within colonial range of your CN work as if capital.''
    \item In particular, this means that if you have a Brazilian CN with sufficient range, you can core inland west african provinces with just an occupation snake.
  \end{itemize}
\end{frame}

\end{document}

