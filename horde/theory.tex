\begin{frame}
  \frametitle{Theory}
  The secret behind pre-1600 WCs is to shift a typical conquest mindset 100
  years back.
  \\
  Similar to the common advice that ``most of your conquests happen
  post-absolutism,'' most horde conquest happens after roughly 1500 -- this
  is when you have your core idea groups and Age of Reformation (AoR) warscore
  (ws) cost reduction age ability.
  \\
  Therefore, the primary objective of a horde player until then is to snake
  around to open up expansion paths, secure a semblance of an economy (mainly
  trade), fill up the core idea groups, and convert to a religion that provides
  core cost reduction (CCR).
\end{frame}

\subsubsection{Idea Groups}
\begin{frame}
  \frametitle{Idea Groups (in order)}
  \begin{enumerate}
    \item \idea{diplo}\footnote{There is no consensus on the exact
        order of the first three idea groups. Take what I say here as my
      opinion fitting the playstyle I am presenting.}
      \begin{itemize}
        \item WS cost, diplomats, and cheaper truce breaking
        \item Diplo points tend to be the most available points in the early
          game; filling out an admin idea group will be too slow at this point.
      \end{itemize}
    \item \idea{humanist}
      \begin{itemize}
        \item Aside from the obvious unrest decreasing buffs, RU gives stab
          cost, and accepted culture slots help a lot.
      \end{itemize}
    \item \idea{admin}
      \begin{itemize}
        \item CCR and governing capacity (GC)\ldots enough said
      \end{itemize}
    \item \idea{explo}
      \begin{itemize}
        \item Required for certain discoveries but mainly for new world
        \item When unlocking new idea group slots, you can temporarily go
          \idea{explo1} to hire explorers and conquistadors.
      \end{itemize}
  \end{enumerate}
\end{frame}
